
% =============================================================================
%                          Chapter 5: Software Testing
% =============================================================================


\chapter*{Chapter 5}              
\addcontentsline{toc}{chapter}{Chapter 5}  
\markboth{Software Testing}{Software Testing}  % header text
\section{Software Testing}

Software Testing is the most crucial part of Software Development Process. It is the investigation or evaluation of a software component, improving them, and finding bugs and defects. Testing is usually done by executing a system in such a way that it identifies any gaps, errors, or missing requirements in contrary to the actual requirements.

\subsection{Testing Methodology}
It is essential to have a testing plan in place to ensure that the product delivered is robust and stable, and is delivered on a predictable timeline.


In this section you will discuss the reason of various testing techniques that you have used to test the software you have created such as integration testing, component testing and system testing etc.

\subsection{Testing Environment}
Describe and discuss the reason to use the selected testing environment.

\subsection{Test Cases}
You should describe how you demonstrated that the system works as intended (or not, as the case may be). Include comprehensible summaries of the results of all critical tests that were carried out. You might not have had the time to carry out any full rigorous tests you may not even got as far as producing a testable system. However, you should try to indicate how confident you are about whatever you have produced, and also suggest what tests would be required to gain further confidence

\subsubsection{Test Case 1}

\begin{itemize}
    \item Test case description
    \item How test case was generated
    \item Expected result of the test case
    \item Actual result of the test case
\end{itemize}

% ----- TABLE -----
\begin{table}[H]
    \centering
    \caption{Test Case Name}
    \label{tab:test_case_1}
    \begin{tabular}{|p{7cm}|p{7cm}|}
        \hline
        Date: 06 June 2017 & \\ \hline
        System: Menu Drive & \\ \hline
        Objective: View location of delivery boy & \textit{Test ID: 1} \\ \hline
        Version: 1 & \textit{Test Type: Unit testing} \\ \hline
        \multicolumn{2}{|p{14cm}|}{Input: \newline Longitude = 33.7294 \newline Latitude = 73.0931} \\ \hline
        \multicolumn{2}{|p{14cm}|}{Expected Result: return Islamabad location.} \\ \hline
        \multicolumn{2}{|p{14cm}|}{Actual Result: passed} \\ \hline
    \end{tabular}
\end{table}





\setcounter{chapter}{5}           
