
% =============================================================================
%                            Chapter 1: Introduction
% =============================================================================


\chapter*{Chapter 1}
\addcontentsline{toc}{chapter}{Chapter 1}
\markboth{Introduction}{Introduction} 
\section{Introduction}
A good introduction should tell the reader what the project is about without assuming special knowledge and without introducing any specific material that might obscure the overview. Introduction is mostly written for readers' who are not experts of the field. Using this template is a great start \cite{logicalhammad}.

\subsection{Project Introduction}
Brief description of the project: Briefly describe what the project is. It explains the immediate context of the problem you want to solve. To start writing your introduction, first come up with a simple one-sentence summary of the goal of your project. Then elaborate the statement briefly.

Introduction of the beneficiary of the project: In this section you will describe the organization for which you intend to build this project. The organization could be a company or a single user who would benefit from your application in a particular way.


\subsection{Existing Examples / Solutions}
In this section you are supposed to provide the survey of the existing systems or products that belong to the same domain that you have identified in the previous section. 
This survey should include history and working details of prominent systems/products that solve the similar problem that you intend to solve. 
You should select a relevant case study carefully and also identify any shortcomings or all those features that are missing in the existing solutions and you intend to provide in your project.

\subsection{Business Scope}
This section will describe the project from the business point of view. 
It will identify its business potential with the target market. 

You can identify the additional potential customers and emphasize on the possibility of deploying this project as a commercial product. 
You can identify the effort and cost of all the additional resources you need in your project to achieve this aim.


\subsection{Useful Tools and Technologies}
This section should mention possible technologies that could be used during the designing, development and testing of your project.

You should mention the technologies that you intend to use with a brief but technical justification for your decision. Your discussion should include

% ----- NUMBERED LIST -----
\begin{enumerate}
    \item What programming language are you using and why?
    \item Which development environment do you intend to use and why?
    \item What database (if any) are you using and why?
    \item Which operating systems will support your software?
    \item What network protocol (if any) is implemented and why?
\end{enumerate}

\subsection{Project Work Break Down}
All projects require planning, including an outline of who on the team is doing what and when; thus, you will need to include a Work Breakdown chart. You must identify all the components of the project and also specify how much time you will spend on each component. The justification should include your strengths and weakness from the project point of view and it should indicate that you have allocated appropriate time period for those modules that you find yourself as your weak points. A typical software project is divided into several parts as shown in Figure \ref{fig:ex3}.

You are supposed to provide work breakdown structure of your project along with the assignment details that which group member will be performing which tasks.

% ----- FIGURE -----
\begin{figure}[H]
    \centering
    % Change width as needed (0.5 - 1.0 is standard)
    \includegraphics[width=0.9\textwidth]{figures/image03.png} 
    \caption{Work breakdown structure}
    \label{fig:ex3}
\end{figure}


\subsection{Project Time Line}
A Gantt chart outlines what aspects of the project will be completed and by when. It is an important component of good project management and something you will probably be asked to do as a part of your job. A sample Gantt chart is shown in Figure \ref{fig:ex4}.

% ----- FIGURE -----
\begin{figure}[H]
    \centering
    % Change width as needed (0.5 - 1.0 is standard)
    \includegraphics[width=0.9\textwidth]{figures/image04.png} 
    \caption{Sample Gantt Chart}
    \label{fig:ex4}
\end{figure}



\setcounter{chapter}{1}